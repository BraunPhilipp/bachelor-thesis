%!TEX root = thesis.tex
% The basic document settings
\documentclass[11pt,a4paper,twoside,abstract=on]{scrreprt}
% If you have problems compiling this template, try adding the following
% package in order to allow for more packages to be included
% A possible error could look like this:
% "No room for a new \dimen \newdimen \MPscratchDim"
%\usepackage{etex}

% Compatibility issues solved
% \usepackage{scrhack}

\usepackage{amsmath}
\usepackage[makeroom]{cancel}
\usepackage[ngerman,english]{babel}
\usepackage{setspace}
\usepackage{subcaption}

\usepackage{sansmath}

\DeclareMathOperator*{\argmax}{argmax}

% Further (Koma) options:
% Table of references and indices in TOC
\KOMAoptions{bibliography = totoc}
\KOMAoptions{index = totoc}
% Behaviour of paragraphs (full = empty line, false = ident)
\KOMAoptions{parskip = full}
% Draw line under page header
\KOMAoptions{headsepline = true}
% New chapters always start on right page
\KOMAoptions{open = right}

% Page formatting
\setlength{\evensidemargin}{0.5cm}
\setlength{\oddsidemargin}{0.5cm}
\setlength{\topmargin}{0.5cm}
\setlength{\textwidth}{15cm}

% Pre formatted pagestyle
\pagestyle{headings}

% Font and language
%\usepackage{german}
%\usepackage[latin1]{inputenc}


%\usepackage[ngerman]{babel}
%\usepackage{lmodern} % Better font
%\usepackage[utf8]{inputenc}
%\usepackage[T1]{fontenc}

\usepackage{ifxetex}

\ifxetex
  \usepackage{fontspec}
  \usepackage{polyglossia}
  \setmainlanguage[spelling=new, latesthyphen=true]{german}
\else
  \usepackage[T1]{fontenc}
  \usepackage[utf8]{inputenc}
  \usepackage[ngerman]{babel}
  \usepackage{lmodern}
\fi

%\definecolor{darkgray}
%\font{Adobe Caslon Pro}
% Easier quotation marks

%\usepackage [autostyle, english = british]{csquotes}
%\MakeOuterQuote{"}

% Mandatory for some bib styles
%\usepackage{natbib}

% Maths:
\usepackage{amsfonts, amsmath, amsthm, amscd, amssymb, array, bm, mathtools}
\usepackage{nicefrac}
\usepackage{cases} % For sub numbered cases
\numberwithin{equation}{section} % Equation umbering chapter.section.equation
\usepackage{bigints} % Big integral signs

% Bring color into Latex
\usepackage[svgnames]{xcolor}
\usepackage{colortbl}
\definecolor{hellgrau}{rgb}{0.9,0.9,0.9}

% Several graphics packages
% General graphics
\usepackage{graphicx}
\usepackage{grffile}
\usepackage{lipsum}
%\usepackage{flafter}
% LaPrint (exports matlab figures for latex)
\usepackage{color, psfrag}
% ps-graphics
\usepackage{pstricks,pst-plot,pst-node}
% TIKZ
\usepackage{tikz}
\usetikzlibrary{arrows,matrix,positioning}
% Set the path for figure files
\graphicspath{{./figures/}}
\def\layersep{2.5cm}
\tikzset{basic/.style={draw,fill=blue!50,text width=1em,text badly centered}}
\tikzset{input/.style={basic,circle}}
\tikzset{weights/.style={basic,rectangle,minimum width=0.7cm, minimum height=0.6cm}}
\tikzset{functions/.style={basic,circle,fill=blue!50}}


% Prevent figures from floating too much
\usepackage[section]{placeins}

% SI-Units
\usepackage{siunitx}

% Algorithms
% \usepackage{algpseudocode,algorithmicx}
% \renewcommand{\algorithmicrequire}{\textbf{Input:}}
%\renewcommand{\algorithmicensure}{\textbf{Output:}}

\usepackage{algorithm,algpseudocode}
% \usepackage[ruled]{algorithm2e}

% Better tabulars, enables thicker lines
\setlength{\doublerulesep}{0pt}
\usepackage{multirow} % Multirows in tabulars
\usepackage{longtable, tabu} % Flexible tabulars i.e. page breaks and horizontal fill
\newtabulinestyle{mydashline=on 1.5pt off 2pt}
\tabulinesep=1mm

% Creating an index
%\usepackage{makeidx}
% Adding items to index more comfortably
%\newcommand{\idx}[1]{#1\index{#1}}

% Abbrevations
% Note that the abbrevations support commands like \ac{acronym} and the creation
% of a list of acronyms. In order to create this list, the document needs to be
% compiled as follows (includes optional bibtex). Use pdflatex or latex as preferred
%     (pdf)latex thesis.tex
%     bibtex thesis.aux
%     makeglossaries thesis
%     (pdf)latex thesis.tex
%\usepackage[nomain, toc, shortcuts]{glossaries}
% Removes a dot at the end of every description
%\renewcommand*{\glspostdescription}{}
%\loadglsentries{acronym.tex}
%\bibliographystyle{alphadin}



% Links in PDFs
\usepackage{hyperref}
\hypersetup{colorlinks=false, pdfborder={0 0 0}}
\makeatletter
\hypersetup{pdftitle = \@title}
\hypersetup{pdfauthor = \@author}
\makeatother
\hypersetup{pdfsubject = \thesistypegerman}

% Automatic referencing (e.g. \ref{...} creates figure 2.1.3)
\usepackage[english]{cleveref}
\renewcommand{\figurename}{Figure}%
% No eq. in front of equations
\crefformat{equation}{#2(#1)#3}
\Crefformat{equation}{#2(#1)#3}

% Some math macros:
% Operators
\newcommand{\modulo}{\ensuremath{\mbox{mod }}}
\newcommand{\abs}[1]{\ensuremath{\left\vert #1 \right\vert}}
\newcommand{\card}[1]{\ensuremath{\left\vert #1 \right\vert}}
\newcommand{\norm}[2]{\ensuremath{\left\Vert #1 \right\Vert_{#2}}}
\newcommand{\divisor}{\ensuremath{\mathrm{div}}}
\newcommand{\ord}{\ensuremath{\mathrm{ord}}}
\newcommand{\supp}{\ensuremath{\mathrm{supp}}}
\newcommand{\ggt}{\ensuremath{\mathrm{ggt}}}
\newcommand{\charac}{\ensuremath{\mathrm{char}}}
\newcommand{\ud}{\ensuremath{\,\mathrm{d}}} % Integral d (dx, dy, etc.)
\newcommand{\ppart}[1]{\ensuremath{\left[#1\right]^{+}}}
\newcommand{\npart}[1]{\ensuremath{\left[#1\right]^{-}}}
\DeclareMathOperator{\diag}{diag}
\newcommand{\vecop}{\ensuremath{\operatorname{\mathbf{vec}}}}

% Sets
\newcommand{\Z}{\ensuremath{\mathbb{Z}}}
\newcommand{\N}{\ensuremath{\mathbb{N}}}
\newcommand{\J}{\ensuremath{\mathbb{J}}}
\newcommand{\HD}{\ensuremath{\mathbb{P}}}
\newcommand{\D}{\ensuremath{\mathbb{D}}}
\newcommand{\R}{\ensuremath{\mathbb{R}}}  % Real numbers
\newcommand{\Rp}{\ensuremath{\R_{\geq 0}}} % Nonnegative real numbers
\newcommand{\CO}{\ensuremath{\mathcal{O}}}
\newcommand{\F}{\ensuremath{\mathbb{F}}}
\newcommand{\I}{\ensuremath{\mathcal{I}}}
\newcommand{\allsubsets}[1]{\mathfrak{B}(#1)}
\newcommand{\X}{\ensuremath{\mathbb{X}}}
\newcommand{\Y}{\ensuremath{\mathbb{Y}}}

% Optimization
\newcommand{\minimize}{\ensuremath{&\operatorname{minimize} &&}}
\newcommand{\minimizex}[1]{\ensuremath{&\underset{#1}{\operatorname{minimize}}&&}}
\newcommand{\maximize}{\ensuremath{&\operatorname{maximize} &&}}
\newcommand{\maximizex}[1]{\ensuremath{&\underset{#1}{\operatorname{maximize}}&&}}
\newcommand{\st}{\ensuremath{&\operatorname{subject~to}&&}}
\newcommand{\stff}{\ensuremath{&&&}}

% For vectors and matrices
\renewcommand{\vec}[1]{\ensuremath{\bm{\MakeLowercase{#1}}}}
\newcommand{\mat}[1]{\ensuremath{\bm{\MakeUppercase{#1}}}}
\newcommand{\vecidx}[2]{\ensuremath{\MakeLowercase{#1}_{#2}}}
\newcommand{\vecidxB}[2]{\ensuremath{\left[#1\right]_{#2}}}
\newcommand{\vecidxBppart}[2]{\ensuremath{\left[#1\right]^{+}_{#2}}}
\newcommand{\vecidxBnpart}[2]{\ensuremath{\left[#1\right]^{-}_{#2}}}
\newcommand{\matidx}[3]{\ensuremath{\MakeLowercase{#1}_{#2 #3}}}
\newcommand{\matidxB}[3]{\ensuremath{\left[#1\right]_{#2 #3}}}
\newcommand{\matidxBppart}[3]{\ensuremath{\left[#1\right]_{#2 #3}^{+}}}
\newcommand{\transp}[1]{\ensuremath{#1^{\top}}}
\newcommand{\inv}[1]{\ensuremath{#1^{-1}}}
\newcommand{\pinv}[1]{\ensuremath{#1^{\dagger}}}
\newcommand{\opt}[1]{\ensuremath{#1^{\star}}}
\newcommand{\unitvec}{\vec{e}}
\newcommand{\identmat}{\mat{i}}
\newcommand{\zerovec}[1]{\vec{0}_{#1}}
\newcommand{\zeromat}[2]{\mat{0}_{#1\times #2}}
\newcommand{\onevec}[1]{\vec{1}_{#1}}
\newcommand{\onemat}[2]{\mat{1}_{#1\times #2}}

% Logics
\newcommand{\logicnot}{\lnot}
\newcommand{\logicand}{\wedge}
\newcommand{\logicor}{\vee}
\newcommand{\logicxor}{\oplus}

% Colors
\newcommand{\bl}{\color{blue}}
\newcommand{\bk}{\color{black}}
\newcommand{\rd}{\color{red}}
\newcommand{\gr}{\color{green}}

\newcommand{\mcl}{\mathcal}

% Convert number 1 to 12 into corresponding months
%\newcommand{\monthword}[1]{\ifcase#1\or January\or February\or March\or April\or
%                                        May\or June\or July\or August\or
%                                        September\or October\or November\or December\fi}

%\makeglossaries   % -> run makeglossaries thesis
\makeindex
