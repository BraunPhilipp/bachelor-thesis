\chapter{Adversarial Attacks}\label{sec:chapter}

\section{Introduction}\label{sec:section}

\begingroup
Misclassification of DNNs have become more prominent in recent years. Especially in
medical applications and automous driving errors of DNN can be fatal \cite{NeuralMed,ControlPipeline}.
To reduce the failure of these systems fast and secure methods to evaluate these networks are necessary.
In the following various optimization techniques are introduced to find networks prone to errors.
To compare different attack characteristics a regression problem is used.
Adversarial attacks (input perturbations for a NN) can be introduced into networks in many ways.
Eykholt et al. \cite{RealWorld} has shown that these attacks can be performed in the real world. The research paper
uses stickers and graffitis to test adversarial attacks.

Generally, the aim of an attacker is to create an additive perturbation $ \bm{\eta} \in \mathbb{R}^M$ with $M \triangleq n^{(1)}$ that is unnoticeable by
an outstanding entity. The loss or cost function for $\bm{x} + \bm{\eta}$ is given by $ \mcl{L}(\bm{x} + \bm{\eta}) = \lvert\lvert f(\bm{x} + \bm{\eta}) - \bm{y} \rvert\rvert^2_2 $.
The maximum input noise is limited through $\lvert\lvert \bm{\eta} \rvert\rvert_p \leq \epsilon$ with $\epsilon \in \mathbb{R}$ and $p \in \mathbb{N}$.
The objective of an attacker is to maximize this loss. Therefore, the problem can be defined as
\endgroup

\begin{equation}
\begin{aligned}
	\max_{ \bm{\eta} } \; \mcl{L}(\bm{x} + \bm{\eta}) = \max_{ \bm{\eta} } \; \lvert\lvert f(\bm{x} + \bm{\eta}) - \bm{y} \rvert\rvert^2_2 \quad s.t. \quad \lvert\lvert \bm{\eta} \rvert\rvert_p  \leq \epsilon \\
\end{aligned}
\label{intromax}
\end{equation}


\begingroup
Unfortunately, this problem cannot be easily solved, since $f(\bm{x}+\bm{\eta})$ depends on $\bm{\eta}$.
\endgroup


\section{Convex Attacks}\label{sec:section}

% Linear %
\subsection{Linear Programming Problem}\label{sec:section}

\begingroup
One of the earliest approaches to solve \ref{intromax} was the Fast Gradient Sign Method (FGSM), which was first introduced by Goodfellow et al. \cite{Goodfellow}. The paper
linearizes the cost function around $\bm{x}$ and sets $p=\infty$. The optimal max-norm is then given by
\endgroup

\begin{equation}
\begin{aligned}
	\bm{\eta} = \epsilon \; \text{sign}( \nabla_{\bm{x}} \mcl{L}) \quad w. \quad \lvert\lvert \bm{\eta} \rvert\rvert_{\infty} \leq \epsilon
\end{aligned}
\label{fastgradientsign}
\end{equation}

\begingroup
This linarization can be performed using a taylor expansion as shown in \ref{taylorlinear}. If $ \mcl{L}(\bm{x}) = \left\lvert\lvert f(\bm{x}) - \bm{y} \right\rvert\rvert^2_2 $
is used the product rule can be applied and the effect of $\bm{\eta}$ thoroughly studied.
\endgroup

\begin{equation}
\begin{aligned}
	\mcl{L}(\bm{x} + \bm{\eta}) &\approx \mcl{L}(\bm{x}) + \bm{\eta}^T \nabla \mcl{L}(\bm{x}) = \mcl{L}(\bm{x}) + \bm{\eta}^T \nabla \mcl{L}(\bm{x}) \\[10pt]
		&= \mcl{L}(\bm{x}) + \bm{\eta}^T \nabla( (\bm{y} - f(\bm{x}))^T (\bm{y} - f(\bm{x})) ) \\[10pt]
		&= \mcl{L}(\bm{x}) + \bm{\eta}^T \nabla( \bm{y}^T\bm{y} - f(\bm{x})^T \bm{y} + f(\bm{x})^T f(\bm{x}) - \bm{y}^T f(\bm{x}) ) \\[10pt]
		&= \mcl{L}(\bm{x}) + \bm{\eta}^T (-\bm{J}_f(\bm{x})^T \bm{y} - \bm{y}^T \bm{J}_f(\bm{x}) + \bm{J}_f(\bm{x}) f(\bm{x}) + f(\bm{x})^T \bm{J}_f(\bm{x})) \\[10pt]
		&= \mcl{L}(\bm{x}) - 2 \bm{\eta}^T \bm{J}_f(\bm{x}) ( \bm{y} - f(\bm{x}) ) \\
\end{aligned}
\label{taylorlinear}
\end{equation}

\begingroup
As shown above this leads to a problem for perfectly fitted models, where $ \bm{y} = f(\bm{x})$ because the gradient $\nabla \mcl{L} = 0$. This
is especially problematic for autoencoders, which add additonal features to the input like colorization. $\bm{y} \in \mathbb{R}^N$ is
usually unknown in these systems and needs to be determined during a first iteration of the DNN. Therefore,
another point $ \bm{\tilde{x}}$ needs be defined, which lies in close proximity to $\bm{x}$. In this point the gradient $ \nabla \mcl{L}(\bm{\tilde{x}}) \neq 0 $
and $ \lvert\lvert x - \tilde{x} \rvert\rvert_p \ll \epsilon $. Therefore, \ref{intromax} can now be generally solved through
\endgroup


\begin{equation}
% \tag{$\dagger$}
\begin{aligned}
	\max_{\bm{\eta}} \nabla \mcl{L}(\bm{\tilde{x}})^T \bm{\eta} \quad s.t. \quad \lvert\lvert \bm{\eta} \rvert\rvert_p \leq \epsilon
\end{aligned}
\label{maxlinopt}
\end{equation}

\begingroup
In the following, $\bm{\tilde{x}}$ and $\bm{x}$ are not further separated.
The closed-form solution for different $p$-norms are given below. Generally, a solution is found using
the upper boundary of the maximization problem.
\endgroup

\begin{equation}
\begin{aligned}
p = 1: & \quad \bm{\eta} = \epsilon \; \bm{e}_{k^\star} \quad w. \quad k^\star = \argmax_{k} \; \lvert (\nabla \mcl{L}(x))_k \rvert \\[20pt]
p = 2: & \quad \bm{\eta} = \epsilon \; \frac{\nabla \mcl{L}(\bm{x})}{\lvert\lvert \nabla \mcl{L}(\bm{x}) \rvert\rvert_2} \\[20pt]
p = \infty: & \quad \bm{\eta} = \epsilon \; \text{sign}(\nabla \mcl{L}(\bm{x}))
\end{aligned}
\label{linear_programming_closed}
\end{equation}


$\bm{Proof.}$ The proof for all linear solutions follows the Hölder inequality \cite{Hoelder}. The theorem states that if $q \triangleq \frac{p}{p-1}$ and $p, q \in [1, \infty]$
the upper boundary can be found by

\begin{equation}
\begin{aligned}
	\lvert\lvert \bm{f}^T\bm{g} \rvert\rvert_1 \leq \lvert\lvert \bm{f} \rvert\rvert_p \lvert\lvert \bm{g} \rvert\rvert_q
\end{aligned}
\label{hoelders_inequality}
\end{equation}

and can be reached through

\begin{equation}
\begin{aligned}
	\sup_{\lvert\lvert \bm{\eta} \rvert\rvert_p \leq 1} \nabla \mcl{L}(\bm{x})^T \bm{\eta} = \lvert\lvert \nabla \mcl{L}(\bm{x}) \rvert\rvert_q
\end{aligned}
\label{linear_programming_sup}
\end{equation}

The upper bound is then reached through

\begin{equation}
\begin{aligned}
	\bm{\eta} = - \epsilon \frac{1}{\lvert\lvert \nabla \mcl{L}(\bm{x}) \rvert\rvert_q^{q-1}} \cdot ( \text{sign}( \nabla \mcl{L}(\bm{x}) ) \odot \lvert \nabla \mcl{L}(\bm{x}) \rvert^{q-1} )
\end{aligned}
\label{linear_programming_solution}
\end{equation}

Since there is no larger supremum, $\bm{\eta}$ needs to be the optimal solution.
Setting $p$ to the respective norm then results in the closed-form solutions in \ref{linear_programming_closed}. $ \blacksquare $

\begingroup
To find even more accurate solutions to the linear programming problem iterative solutions can be used. One of the first attempts was the
DeepFool algorithm \cite{DeepFool}. After calculating an adversarial example for a particular input vector, a new noise vector is calculated
using the previous adversarial input. For example, the iterative solution for the $l_{\infty}$-norm
is calculated using the following pseudocode. Iterative solutions for other norms follow the same procedure.
\endgroup

%%% PSEUDO CODE

\begin{algorithm}
\caption{Iterative Solution for $p = \infty$ (Linear Programming Problem)}
\label{alg:euclid}
\begin{algorithmic}[1]
\State \textbf{input:} $\; \bm{x}, f(\bm{x}), \bm{\eta}, N$
\State \textbf{output:} $\; \bm{\eta}$
\State \textbf{initialize:} $\bm{\tilde{x}} \gets \bm{x}$
\For{$i \gets 1, ..., N$}
	\State $ \bm{\eta} \gets \frac{\bm{\epsilon}}{N} \; \text{sign}(\nabla \mcl{L}(\bm{\tilde{x}})) $
	\State $ \bm{\tilde{x}} \gets \bm{\tilde{x}} + \bm{\eta} $
\EndFor
\end{algorithmic}
\end{algorithm}


% Quadratic %

\subsection{Quadratic Programming Problem}\label{sec:section}


Similar to the loss function,
$ f(\bm{x}) $ can be generally regarded as a non-convex and non-linear function. But since $ \lvert\lvert \bm{\eta} \rvert\rvert_p \leq \epsilon $,
the function can be approximated using a taylor expansion in close proximity to $\bm{x}$. This results in the following
quadratic programming problem.


\begin{equation}
\begin{aligned}
\mcl{L}(\bm{x} + \bm{\eta}) &= \lvert\lvert \bm{y} - f(\bm{x} + \bm{\eta}) \rvert\rvert^2_2 \\[10pt]
&= (\bm{y} - f(\bm{x} + \bm{\eta}))^T(\bm{y} - f(\bm{x} + \bm{\eta})) \\[10pt]
&= \bm{y}^T\bm{y} - f(\bm{x} + \bm{\eta})^T\bm{y} - \bm{y}^T f(\bm{x} + \bm{\eta}) + f(\bm{x} + \bm{\eta})^Tf(\bm{x} + \bm{\eta}) \\[10pt]
&= \lvert\lvert \bm{y} \rvert\rvert^2_2 - 2 \bm{y}^T f(\bm{x} + \bm{\eta}) + \lvert\lvert f(\bm{x} + \bm{\eta}) \rvert\rvert^2_2 \\[10pt]
&\approx \lvert\lvert \bm{y} \rvert\rvert^2_2 - 2 \bm{y}^T (f(\bm{x}) + \bm{J}_f(\bm{x})\bm{\eta}) + \lvert\lvert f(\bm{x}) + \bm{J}_f(\bm{x}) \bm{\eta} \rvert\rvert^2_2 \\[10pt]
&= \lvert\lvert \bm{y} \rvert\rvert^2_2 - 2 \bm{y}^T (f(\bm{x}) + \bm{J}_f(\bm{x})\bm{\eta}) + (f(\bm{x}) + \bm{J}_f(\bm{x})\bm{\eta})^T(f(\bm{x}) + \bm{J}_f(\bm{x})\bm{\eta}) \\[10pt]
&= \cancel{ \lvert\lvert \bm{y} \rvert\rvert^2_2 } - 2 \bm{y}^T ( \cancel{ f(\bm{x}) } + \bm{J}_f(\bm{x})\bm{\eta})) + \cancel{  \lvert\lvert f(\bm{x}) \rvert\rvert^2_2 } + 2 \bm{\eta}^T \bm{J}_f(\bm{x})^Tf(\bm{x}) + \lvert\lvert \bm{J}_f(\bm{x}+\bm{\eta}) \rvert\rvert^2_2 \\[10pt]
&= 2 \bm{\eta}^T \bm{J}_f(\bm{x})^T (f(\bm{x}) - \bm{y}) + \lvert\lvert \bm{J}_f(\bm{x}) \bm{\eta} \rvert\rvert^2_2 \\
\end{aligned}
\label{taylorquadratic}
\end{equation}

\begingroup
The simplified convex problem for $\mcl{L}$ is further simplified by using $\bm{y} \approx f(\bm{x})$.
Therefore, the problem gets simpliefied to the following form.
\endgroup

\begin{equation}
\begin{aligned}
	\max_{\bm{\eta}} \; \lvert\lvert \bm{J}_f(\bm{x}) \bm{\eta} \rvert\rvert_2^2 \quad s.t. \quad \lvert\lvert \bm{\eta} \rvert\rvert_p \leq \epsilon
\end{aligned}
\label{quadintro}
\end{equation}

\begingroup
This problem leads to a convex quadratic bowl maximization under a $l_p$-ball constraint. The maximization problem \ref{quadintro} can be reformulated using
\endgroup

\begin{equation}
\begin{aligned}
	\lvert\lvert \bm{J}_f(\bm{x}) \bm{\eta} \rvert\rvert_2 = \epsilon \; \lvert\lvert \bm{J}_f(\bm{x}) \frac{\bm{\eta}}{\epsilon} \rvert\rvert_2
\end{aligned}
\label{quadfollow}
\end{equation}

\begingroup
Due to $\lvert\lvert \frac{\bm{\eta}}{\epsilon} \rvert\rvert_p \leq 1$, the problem can be solved using operator norms. If
$\bm{M}: V \rightarrow W$ is a linear transformation the operator norm is given by the subsequent term.
\endgroup

\begin{equation}
\begin{aligned}
	\lvert\lvert \bm{M} \rvert\rvert_{V \rightarrow W} = \sup_{\bm{v} \in V \backslash \{0\}} \frac{\lvert\lvert \bm{M v} \rvert\rvert_W}{\lvert\lvert \bm{v} \rvert\rvert_V} = \sup_{\lvert\lvert \bm{v} \rvert\rvert_V \leq 1} \lvert\lvert \bm{M v}  \rvert\rvert_W
\end{aligned}
\label{operatornorm}
\end{equation}

\begingroup
Since $\lvert\lvert \bm{\eta} \rvert\rvert_p = \epsilon$ is the upper bound, \ref{quadfollow} can be estimated through $\epsilon \; \lvert\lvert \bm{J}_f(\bm{x}) \frac{\bm{\eta}}{\epsilon} \rvert\rvert_2 \leq \epsilon \; \lvert\lvert \bm{J}_f(\bm{x}) \rvert\rvert_{p \rightarrow 2}$.
For $l_1$ and $l_2$ this operator norm has a closed-form solution.

For $l_2$ a solution is provided by a theorem from Ky Fan, 1950 \cite{KyFan, BigData}. If $\bm{M} \in \mathbb{R}^{n \times n}$ is a symmetric matrix with sorted
eigenvalues $\lambda_n (\bm{M}) \leq ... \leq \lambda_1 (\bm{M})$ and $\bm{v} \in \mathbb{R}^n$ a vector. The solution is given by
\endgroup

%% kyfan theorem l2 optimization

\begin{equation}
\begin{aligned}
	\max_{\lvert\lvert \bm{v} \rvert\rvert = 1, \bm{v} \in \mathbb{R}^n} \bm{v^T M v} = \lambda_{max} (\bm{M})
\end{aligned}
\label{kyfanmax}
\end{equation}


\begingroup
Therefore, to maximize \ref{quadfollow} the eigenvector $\bm{v}_{max}$ of $\bm{J}_f(\bm{x})^T \bm{J}_f(\bm{x})$ needs to be found, which corresponds to
the largest eigenvalue. Then, the optimal solution to \ref{quadintro} is given by $\bm{\eta} = \pm \epsilon \; \bm{v}_{max}$ \cite{BigData}.
\endgroup


%%%% PROOF KY FAN

\begingroup
$\bm{Proof.}$ Suppose that $\bm{u}_1, ..., \bm{u}_n$ are eigenvectors of $\bm{M}$ corresponding to $\lambda_1(\bm{M}) \geq ... \geq \lambda_n(\bm{M})$
and $\bm{U} = [\bm{u}_1, ..., \bm{u}_n]$. Every vector $\bm{v}$ can be written as $\bm{v} = a_1 \bm{u}_1 + ... + a_n \bm{u}_n = \bm{U} \bm{a}$ where
$\bm{a} = [a_1, ..., a_n]^T$. We get to
\endgroup

\begin{equation}
\begin{aligned}
	\max_{\bm{v} \in \mathbb{R}^n, \bm{v}^T\bm{v}=1} \bm{v}^T \bm{M} \bm{v} &= \max_{\bm{a} \in \mathbb{R}^n, \bm{a}^T\bm{a}=1} \bm{a}^T\bm{U}^T \bm{M} \bm{U}\bm{a} \\
	&=  \max_{\bm{a} \in \mathbb{R}^n, \bm{a}^T\bm{a}=1} \sum_{i=1}^n \lambda_i (\bm{M}) a_i^2 \leq \lambda_1 (\bm{M})
\end{aligned}
\label{kyfan_proof}
\end{equation}

$\blacksquare$

%% l1 closed-form solution

\begingroup
For $p = 1$ a closed-form solution is provided by \ref{l1_norm_solution}. It can be assumed that the conditions from \ref{operatornorm} still apply.
\endgroup

\begin{equation}
\begin{aligned}
	\lvert\lvert \bm{M} \rvert\rvert_{1 \rightarrow 2} = \max_{k \in \{1, ..., n\}} \lvert\lvert \bm{m}_k \rvert\rvert_2
\end{aligned}
\label{l1_norm_solution}
\end{equation}

\begingroup
$\bm{m}_k$ defines the $k$-th column vector of the matrix $\bm{M}$. This closed-form solution leads to
\endgroup

\begin{equation}
\begin{aligned}
	\epsilon \; \lvert\lvert \bm{J}_f(\bm{x}) \rvert\rvert_{1 \rightarrow 2} \leq \max_{k \in \{1, ..., n\}} \lvert\lvert \bm{J}_f(\bm{x})_k \rvert\rvert_2
\end{aligned}
\label{l1solution}
\end{equation}

\begingroup
and
\endgroup

\begin{equation}
\begin{aligned}
	\bm{\eta} = \pm \epsilon \; \bm{e}_{k} \quad s.t. \quad k = \argmax_{k \in \{1, ..., n\}} \lvert\lvert \bm{J}_f(\bm{x}) \rvert\rvert_2
\end{aligned}
\label{l1final}
\end{equation}


%%%%% Proof l1
$\bm{Proof.}$ It can be assumed that $\lvert\lvert \bm{v} \rvert\rvert_1 \leq 1$ and $\bm{M} \in \mathbb{R}^{n \times n}$. The column with
the largest norm $\lvert\lvert \bm{m}_k \rvert\rvert_2$ is defined as $\bm{m}_{\alpha}$ with the corresponding scalar $\alpha$. The
proof now follows an iterative procedure of finding an additonal column $\bm{m}_{\beta}$ that improves the solution.


\begin{equation}
\begin{aligned}
	\lvert\lvert \bm{M v} \rvert\rvert_2^2 &= \lvert\lvert \alpha \bm{m}_{\alpha} + \beta \bm{m}_{\beta} \rvert\rvert_2^2
	&\leq \alpha^2 \lvert\lvert \bm{m}_{\alpha} \rvert\rvert_2^2 + \beta^2 \lvert\lvert \bm{m}_{\beta} \rvert\rvert_2^2 \leq \lvert\lvert \bm{m}_{\alpha} \rvert\rvert_2^2
\end{aligned}
\label{l1_proof}
\end{equation}

$\blacksquare$

\begingroup
Intuitively, $\bm{\eta}$ needs to point into the same direction of the maximum of $\bm{J}_f(\bm{x})$ to optimally solve the maximization problem.
For the quadratic programming problem, the closed-form solutions are provided below in a compact form. For $p = \infty$
the problem is unfornuately NP-hard and can therefore not be solved without iterating through all the possible combinations. An
overview of solutions to an operator norm problem is given by \autoref{operatorcompu}.
\endgroup

\begin{equation}
\begin{aligned}
p = 1: \quad &\bm{\eta} = \pm \epsilon \; \bm{e}_{k} \quad s.t. \quad k = \argmax_{k \in \{1, ..., n\}} \lvert\lvert \bm{J}_f(\bm{x}) \rvert\rvert_2 \\[20pt]
p = 2: \quad &\bm{\eta} = \pm \epsilon \; \bm{v}_{max}
\end{aligned}
\label{quadraticclosedform}
\end{equation}

\begin{table}[]
\centering
\def\arraystretch{1.5}
\begin{tabular}{|c|c|c|c|c|}
\hline
                                 & \multicolumn{4}{c|}{\textbf{Co-Domain}}                                                                                                                                                                                       \\ \hline
                                 &      & $l_1$                                                                    & $l_2$                                                                    & $l_\infty$                                                       \\ \hline
 															 	 & $l_1$   & \begin{tabular}[c]{@{}c@{}}Maximum $l_1$-norm\\ of a column\end{tabular} & \begin{tabular}[c]{@{}c@{}}Maximum $l_2$-norm\\ of a column\end{tabular} & \begin{tabular}[c]{@{}c@{}}Maximum $l_\infty$-norm\\ of a column\end{tabular} \\ \cline{2-5}
\multirow{1}{*}{\textbf{Domain}} & $l_2$   & NP-hard                                                               & \begin{tabular}[c]{@{}c@{}}Maximum singular\\ value\end{tabular}      & \begin{tabular}[c]{@{}c@{}}Maximum $l_2$-norm\\ of a row\end{tabular}      \\ \cline{2-5}
                                 & $l_\infty$ & NP-hard                                                            & NP-hard                                                               & \begin{tabular}[c]{@{}c@{}}Maximum $l_1$-norm\\ of a row\end{tabular}      \\ \hline
\end{tabular}

\caption{Operator Norm Computability}
\label{operatorcompu}

\end{table}

%% Single Pixel attacks %%

\section{Single Subset Attacks}\label{sec:section}

\begingroup
Single Pixel Attacks like \cite{SingleClassification} have proven to be successful in fooling various DNNs
for classification tasks. The attacker's goal in this scenario is to find a single pixel, which can be modified,
to create a highly disturbed output. The definition of this attack varies in various research
papers. For instance, Su et al. \cite{SingleClassification} suggests to modifiy only one color channel and to set
other dimensions to zero. In the following, we will attempt to generalize this approach using a greedy approach. The entire
set of pixel information $\mcl{M} = \{1, ..., M\}$ can be devided into $\mcl{S} = \{\mcl{S}_1, ..., \mcl{S}_S\}$
subsets. Each subset represents the color channels from one pixel. The size is determined through $Z = \frac{M}{S}$. Pixels in
a subset are described using $\mcl{S}_S = \{i^1_S, ..., i^Z_S\}$. To count the number of changed subsets, the
$\lvert\lvert \; \cdot \; \rvert\rvert_{0, \mcl{S}}$ zero-$\mcl{S}$ norm is introduced. For $\lvert\lvert \bm{\eta} \rvert\rvert_{0, \mcl{S}}$ this norm counts the number of modified channels.
If an attacker only attempts to change a single pixel the problem can be formulated in the following way:
\endgroup

\begin{equation}
\begin{aligned}
	\max_{\bm{\eta}} \mcl{L}(\bm{x} + \bm{\eta}) \quad s.t. \quad \lvert\lvert \bm{\eta} \rvert\rvert_{\infty} \leq \epsilon \;, \; \lvert\lvert \bm{\eta} \rvert\rvert_{0, \mcl{S}} = 1
\end{aligned}
\label{singlesubsetgeneral}
\end{equation}


\subsection{Single Subset Attack for the Linear Problem}\label{06_single_linear}

\begingroup
As shown in \ref{maxlinopt} the maximization problem in \ref{singlesubsetgeneral} can be approximated using a taylor expansion. The
resulting linear programming problem is shown in \ref{singlesubsetlinear}. Since only one subset is supposed to be modified the
problem can be rewritten as \ref{singlesubsetrewrite}.
\endgroup

\begin{equation}
\begin{aligned}
	\max_{\bm{\eta}} \nabla \mcl{L}(\bm{x})^T \bm{\eta} \quad s.t. \quad \lvert\lvert \bm{\eta} \rvert\rvert_{\infty} \leq \epsilon \;, \; \lvert\lvert \bm{\eta} \rvert\rvert_{0, \mcl{S}} = 1
\end{aligned}
\label{singlesubsetlinear}
\end{equation}

\begin{equation}
\begin{aligned}
	\bm{\eta_{s}} = \argmax_{\bm{\eta}} \nabla \mcl{L}(\bm{x})^T \bm{\eta} \quad s.t. \quad \lvert\lvert \bm{\eta} \rvert\rvert_{\infty} \leq \epsilon \;, \; (\bm{\eta})_{i_s^z} = 0 \; \forall i_s^z \not\in \mcl{S}_{\mcl{S}}
\end{aligned}
\label{singlesubsetrewrite}
\end{equation}

\begingroup
This problem is maximized by choosing a subset for $\bm{\eta}$ that has the largest norm $\lvert\lvert \nabla \mcl{L}(\bm{x}) \rvert\rvert_{\infty} $.
Afterwards, $\bm{\eta}$ is determined by the direction of $\nabla \mcl{L}(\bm{x})$ in the chosen subset. The problem is further constrained by
$\lvert\lvert \bm{\eta} \rvert\rvert_{\infty} \leq \epsilon$. Therefore, the optimal solution lies on the edges of
the hypercube $\lvert\lvert \bm{\eta} \rvert\rvert_{\infty} = \epsilon$. This geometric interpretation leads to the following closed-form solution.
\endgroup


\begin{equation}
\begin{aligned}
	\bm{\eta^{*}} = \bm{\eta_{s^*}} \quad w. \quad s^{*} = \argmax_s \sum_{z=1}^{Z} \lvert \nabla (\mcl{L}(\bm{x}))_{i^z_s} \rvert \; , \; \bm{\eta}_s = \epsilon \sum^{Z}_{z=1} \text{sign}((\nabla \mcl{L}(\bm{x}))_{i_s^z}) \bm{e}_{i^z_s}
\end{aligned}
\label{singlelinearsolution}
\end{equation}

\begingroup
$\bm{\eta_s}$ determines the optimal subset perturbation. A single subset is then selected based on the maximum change to the cost function $\mcl{L}$.
The pseudocode for the linear subset attack is provided by Algorithm 2. For multiple pixels the routine can be run iteratively with $\bm{P}$
holding all used pixels after each iteration.
\endgroup


\begin{algorithm}[ht]
\caption{Single Subset Attack (Linear Solution)}
\label{alg:single_linear}
\begin{algorithmic}[1]
\State \textbf{input:} $h_{opt}, w_{opt}, \theta$
\State \textbf{output:} $\bm{P}, \bm{\eta}_{opt}$
\For{$h \gets 1, ..., H$}
	\For{$w \gets 1, ..., W$}
		\State $\phi_s \gets \sum_{z=1}^Z \lvert \nabla \mcl{L}_{i_s^z} \rvert$
		\If{$\phi_s > \theta \; and \; (h, w) \not\in \bm{P}$}
			\State $h_{opt}, w_{opt}, \bm{\eta}_{opt} \gets h, w, \epsilon \cdot \text{sign}(\nabla \mcl{L}_s)$
			\State $\theta \gets \phi_s$
		\EndIf
	\EndFor
\EndFor
\State $\bm{P} \gets (h_{opt}, w_{opt})$
\end{algorithmic}
\end{algorithm}


\subsection{Single Subset Attack for the Quadratic Problem}\label{06_single_quadratic}


\begingroup
In the same manner as described in \autoref{06_single_linear}, a solution can be found for the quadratic problem.
First of all, the initial problem \ref{06_single_quadratic_problem} is simplified by fixing the subset modified by an attacker
again to $\bm{\eta}_s$.
\endgroup


\begin{equation}
\begin{aligned}
	\max_{\bm{\eta}} \lvert\lvert \bm{J}_f(\bm{x}) \bm{\eta} \rvert\rvert_2^2 \quad s.t. \quad \lvert\lvert \bm{\eta} \rvert\rvert_{\infty} \leq \epsilon \; , \; \lvert\lvert \bm{\eta} \rvert\rvert_{0,\mcl{S}} = 1
\end{aligned}
\label{06_single_quadratic_problem}
\end{equation}

\begin{equation}
\begin{aligned}
	\bm{\eta}_s = \argmax_{\eta} \lvert\lvert \bm{J}_f(\bm{x}) \bm{\eta} \rvert\rvert_2^2 \quad s.t. \quad \lvert\lvert \bm{\eta} \rvert\rvert_{\infty} \leq \epsilon \;, \; (\bm{\eta})_{i_s^z} = 0 \; \forall i_s^z \not\in \mcl{S}_{\mcl{S}}
\end{aligned}
\label{06_single_quadratic_problem_rewrite}
\end{equation}

\begingroup
In contrast to the linear subset attack, there is not a closed-form solution for the quadratic problem.
Instead it is possible to reduce the complexity by finding a suboptimal solution.

Geometrically, \ref{06_single_quadratic_problem_rewrite} can be viewed
as the maximization of a quadratic bowl over a hypercube. By limiting the solution space, we get $\bm{\eta}_s = \epsilon \sum_{z=1}^{Z} \rho_{i_s^z}^* \bm{e}_{i^z_s}$
with $\bm{\rho^*_s} = (\rho_{i_s^1}, ..., \rho_{i_s^z})^T \in \{ -1, +1\}^Z $. Therefore, foreach subset the problem gets rewritten to \ref{06_single_quadratic_solution_rho_1} and \ref{06_single_quadratic_solution_rho_2}.
\endgroup

\begin{equation}
\begin{aligned}
	\bm{\eta}^* = \bm{\eta}_{s*} \quad w. \quad s^* = \argmax_s \lvert\lvert \bm{J}_f(\bm{x}) \bm{\eta}_s \rvert\rvert_2^2 \; , \; \bm{\eta}_s = \epsilon \sum_{z=1}^{Z} \rho_{i_s^z}^* \bm{e}_{i^z_s}
\end{aligned}
\label{06_single_quadratic_solution_rho_1}
\end{equation}

\begin{equation}
\begin{aligned}
	\bm{\rho}_s^* = \argmax_{\rho_s \in \{-1, +1\}^z} \lvert\lvert \bm{J}_f(\bm{x})(\epsilon \sum_{z=1}^{Z} \rho_{i_s^z} \bm{e}_{i_s^z}) \rvert\rvert_2^2 = \argmax_{\bm{\rho}_s \in \{-1, +1\}^Z} \sum_{z=1}^Z \sum_{w=1}^Z \rho_{i_s^z} \rho_{i_s^w} \bm{J}_{i_s^z}^T \bm{J}_{i_s^w}
\end{aligned}
\label{06_single_quadratic_solution_rho_2}
\end{equation}

\begingroup
To find a solution to this NP-hard problem, $\bm{\eta}_s^*$ can be approximated using a greedy method. This is achieved by sorting $\lvert\lvert \bm{J}_{i_s^1} \rvert\rvert_2 \geq ... \geq \lvert\lvert \bm{J}_{i_s^z} \rvert\rvert_2$ and setting $\bm{\rho}_{i_s^1}^* = 1$.
Afterwards, all $\rho_{i_s^z}^*$ are calculated recursively in \ref{06_single_quadratic_recursive} using $\rho_{i_1^s}^*$ as a starting point.
\endgroup

\begin{equation}
\begin{aligned}
	\rho_{i_s^z}^* = \text{sign}((\sum_{j=1}^{z-1} \rho_{i_s^j}^* \bm{J}_{i_s^z})^T \bm{J}_{i_s^z}) \quad \forall z = 2, ..., Z
\end{aligned}
\label{06_single_quadratic_recursive}
\end{equation}


\begingroup
The entire iterative procedure is similar to the solution of Goemans and Williamson to the MacCut problem \cite{MaxCut}. Therefore, the given solution is also limited by the same approximation factor $R = 0.87856$ with $\lvert ALG \rvert \geq R \cdot \lvert OPT \rvert$.
For an image with three color channels ($Z=3$), $\rho_{i_s^z}^*$ looks the following way.
\endgroup

\begin{equation}
\begin{aligned}
z = 1: \quad & \rho_{i_s^1}^* = 1 \\[10pt]
z = 2: \quad & \rho_{i_s^2}^* = \text{sign}( (\rho_{i_s^1}^* \bm{J}_{i_s^1})^T \bm{J}_{i_s^2} ) \\[10pt]
z = 3: \quad & \rho_{i_s^3}^* = \text{sign}( (\rho_{i_s^1}^* \bm{J}_{i_s^1} + \rho_{i_s^2}^* \bm{J}_{i_s^2})^T \bm{J}_{i_s^3} ) \\[10pt]
\end{aligned}
\label{rho_example_z_3}
\end{equation}


\begin{algorithm}[ht]

\caption{Single Subset Attack (Quadratic Solution)}
\label{alg:single_quadratic}
\begin{algorithmic}[1]
\State $\bm{P}, \bm{J}_{norm} \gets None, 0$
\For{$h \gets 1, ..., H$}
	\For{$w \gets 1, ..., W$}
		\For{$d \gets 1, ..., D$}
			\State $n_d \gets \lvert\lvert \bm{J}_{i_s^d} \rvert\rvert_2$
		\EndFor
		\State $\rho_{i_s^1}^*, \hat{\bm{J}}, \bm{I} \gets 1, \bm{J}_{i_s^1}, reverse(argsort(n_1, ..., n_D))$
		\If{$D > 0$}
			\For{$k \gets 2, ..., D$}
				\State $\rho_{i_s^k}^* \gets \text{sign}( (\sum_{j=1}^{k-1} \rho_{i_s^{\bm{I}_k}}^* \bm{J}_{i_s^j})^T \bm{J}_{i_s^{\bm{I}_k}} )$
				\State $\hat{\bm{J}} \gets \hat{\bm{J}} + \rho_{i_s^{\bm{I}_k}}^* \cdot \bm{J}_{i_s^{\bm{I}_k}}$
			\EndFor
		\EndIf
		\If{$\lvert\lvert \sum_{j=1}^{k} \rho_{i_s^k}^* \bm{J}_{i_s^j} \rvert\rvert_2 > \bm{J}_{norm}$ and $(h, w) \not\in \bm{P}$}
			\State $h_{opt}, w_{opt}, \rho_{opt} \gets h, w, \rho^*$
			\State $\bm{J}_{norm} \gets \lvert\lvert \sum_{j=1}^{k} \rho_{i_s^k}^* \bm{J}_{i_s^j} \rvert\rvert_2$
		\EndIf
	\EndFor
\EndFor
\State $\bm{P}, \bm{\eta}_s^* \gets (h_{opt}, w_{opt}), \epsilon \cdot \rho_{opt}$

\end{algorithmic}
\end{algorithm}
